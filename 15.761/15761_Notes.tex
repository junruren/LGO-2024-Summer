\documentclass[11pt]{article}

\usepackage{sectsty}
\usepackage{graphicx}
\usepackage{amsmath,amsthm,amsfonts,amssymb}

% Margins
\topmargin=-0.45in
\evensidemargin=0in
\oddsidemargin=0in
\textwidth=6.5in
\textheight=9.0in
\headsep=0.25in

\title{ 15.761 Introduction to Operations Management }
\author{ Junru Ren }
\date{\today}

\begin{document}
\maketitle	

\section{Process Flow Analysis}

\subsection{Throughput time}

The time that elapses from when the job starts the process to the time it ends the process.

\subsection{Little's Law}

The only law in operations management.

\begin{equation}
    L = \lambda W
\end{equation}

\begin{itemize}
    \item $L$: average number of jobs in system
    \item $\lambda$: average job arrival rate
    \item $W$: average throughput time / average time each job spend in the process
\end{itemize}

\textbf{Caution}: watch out for unit conversion, especially time-related units.


\section{Capacity}

\begin{equation}
    \text{capacity utilization} = \frac{\text{capacity required}}{\text{capacity available}}
\end{equation}

Three ways to change capacity utilization:

\begin{enumerate}
    \item Increase resources: maintain speed but add time available or add resources
    \item Work faster: in same amount of time
    \item Shift demand
\end{enumerate}

If we need to distinct peak and non-peak times instead of just looking at the average behavior, simply break the time
interval into two time intervals: peak and non-peak.

\subsection{Capacity measured in terms of units}

\begin{equation}
    \text{capacity required} = \text{\# of jobs}
\end{equation}

\begin{equation}
    \text{capacity available} = \frac{\text{time available}}{\text{cycle time}}
\end{equation}

\subsection{Capacity measured in terms of time}

\begin{equation}
    \text{capacity required} = (\text{\# of jobs})(\text{cycle time})
\end{equation}

\begin{equation}
    \text{capacity available} = \text{time available}
\end{equation}

\subsection{Adjusted for start-up}

\begin{equation}
    \text{capacity available} = \frac{\text{time available} - \text{throughput time}}{\text{cycle time}} + 1
\end{equation}

If looking for the number of ``whole'' units can be made, round \textbf{down} to the nearest whole number.

\section{Congestion Analysis}

\subsection{Deterministic variability - inventory buildup diagrams}

The balance equation:

\begin{equation}
    \begin{split}
    \left(\text{\# of jobs in system at end of period}\right) =& \left(\text{\# of jobs \textbf{in} system at start of period}\right) \\
    &+ \left(\text{\# of jobs \textbf{arriving} to system during period}\right) \\
    &- \left(\text{\# of jobs serviced (departed system) during period}\right)
    \end{split}
\end{equation}

Can replace ``system'' with ``queue'' or ``service''.

\begin{equation}
    \text{buildup rate} = \text{arrival rate} - \text{departure rate}
\end{equation}

Buildup rate can be negative if there is inventory being worked off. Otherwise, the departure rate cannot exceed the
arrival rate.

\begin{equation}
    \text{average inventory} = \frac{\text{\textbf{area} under the inventory buildup curve}}{\text{total time interval}}
\end{equation}

Also applicable to average queue size if the ``inventory'' is considered as jobs waiting in a queue.

\section{Queuing Analysis}

\subsection{Setup}

\begin{itemize}
    \item $A$: time between successive job arrivals (a.k.a ``interarrival time'')
    \item TODO
\end{itemize}

\subsection{Capacity utilization $\rho$}

\begin{equation}
    \rho = \frac{\lambda}{N \mu}
\end{equation}

\begin{itemize}
    \item $\lambda$: job arrival rate
    \item $N$: number of servers
    \item $\mu$: a server's expected service rate, which is the inverse of the average service time $\bar{S}$ \begin{equation}
        \mu = \frac{1}{\bar{S}}
    \end{equation}
\end{itemize}

\subsection{Coefficient of variation of the interarrival time $CV_S$}

\begin{equation}
    CV_S = TODO
\end{equation}

\subsection{Coefficient of variation of the service time $CV_A$}

\begin{equation}
    CV_A = TODO
\end{equation}

\subsection{Expected number of jobs in the queue $L_q$}

\begin{equation}
    L_q = \frac{\rho^{\sqrt{2(N + 1)}}}{1 - \rho} \frac{C_A^2 + C_S^2}{2}
\end{equation}

\end{document}