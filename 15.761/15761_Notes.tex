\documentclass[11pt]{article}

\usepackage{sectsty}
\usepackage{graphicx}
\usepackage{amsmath,amsthm,amsfonts,amssymb}

% Margins
\topmargin=-0.45in
\evensidemargin=0in
\oddsidemargin=0in
\textwidth=6.5in
\textheight=9.0in
\headsep=0.25in

\title{ 15.761 Introduction to Operations Management }
\author{ Junru Ren }
\date{\today}

\begin{document}
\maketitle	

\section{Process Flow Analysis}

\subsection{Process flow diagram}

Attention!
\begin{itemize}
    \item Always a good idea to sketch out a diagram even if not asked for
    \item Label the resources. e.g. cashier / barista / Christian / Marisa with dashed boxes
    \item Label all the final outcomes
    \item Be consistent with the time unit: hours vs minutes, stick to one
    \item Differentiate ``average rate'' vs ``average time duration'' provided for each step: one is the inverse of another
    \item You can depart more than the arrival: \begin{itemize}
        \item Arrival rate of a later step is $min\left[\lambda_{\text{previous step}}, \left(N_{\text{previous step}} \cdot \mu_{\text{previous step}}\right)\right]$
        \item Depart rate of the current step is $min\left[\lambda, N \cdot \mu\right]$
    \end{itemize}
    \item When interpreting a drawn diagram be \textbf{careful with \fbox{change of the ``perspective''.}} e.g. diagram may reflect the flow of a patient but we may need to change to a doctor's perspective
\end{itemize}

\subsection{Throughput time}

The time that elapses from when the job starts the process to the time it ends the process.

\subsection{Little's Law}

The only law in operations management.

\begin{equation}
    L = \lambda W
\end{equation}

\begin{itemize}
    \item $L$: average number of jobs in system
    \item $\lambda$: average job arrival rate
    \item $W$: average throughput time / average time each job spend in the process
\end{itemize}

\textbf{Caution}: watch out for unit conversion, especially time-related units.


\section{Capacity}

\begin{equation}
    \text{capacity utilization} = \frac{\text{capacity required}}{\text{capacity available}}
\end{equation}

Three ways to change capacity utilization:

\begin{enumerate}
    \item Increase resources: maintain speed but add time available or add resources
    \item Work faster: in same amount of time
    \item Shift demand
\end{enumerate}

If we need to distinct peak and non-peak times instead of just looking at the average behavior, simply break the time
interval into two time intervals: peak and non-peak.

\subsection{Capacity measured in terms of units}

\begin{equation}
    \text{capacity required} = \text{\# of jobs}
\end{equation}

\begin{equation}
    \text{capacity available} = \frac{\text{time available}}{\text{cycle time}}
\end{equation}

\subsection{Capacity measured in terms of time}

\begin{equation}
    \text{capacity required} = (\text{\# of jobs})(\text{cycle time})
\end{equation}

\begin{equation}
    \text{capacity available} = \text{time available}
\end{equation}

\subsection{Adjusted for start-up}

\begin{equation}
    \text{capacity available} = \frac{\text{time available} - \text{throughput time}}{\text{cycle time}} + 1
\end{equation}

If looking for the number of ``whole'' units can be made, round \textbf{down} to the nearest whole number.

\section{Congestion Analysis}

\subsection{Deterministic variability - inventory buildup diagrams}

The balance equation:

\begin{equation}
    \begin{split}
    \left(\text{\# of jobs in system at end of period}\right) =& \left(\text{\# of jobs \textbf{in} system at start of period}\right) \\
    &+ \left(\text{\# of jobs \textbf{arriving} to system during period}\right) \\
    &- \left(\text{\# of jobs serviced (departed system) during period}\right)
    \end{split}
\end{equation}

Can replace ``system'' with ``queue'' or ``service''.

\begin{equation}
    \text{buildup rate} = \text{arrival rate} - \text{departure rate}
\end{equation}

Buildup rate can be negative if there is inventory being worked off. Otherwise, the departure rate cannot exceed the
arrival rate.

\begin{equation}
    \text{average inventory} = \frac{\text{\textbf{area} under the inventory buildup curve}}{\text{total time interval}}
\end{equation}

Also applicable to average queue size if the ``inventory'' is considered as jobs waiting in a queue.

\section{Queuing Analysis}

\subsection{Setup}

\begin{itemize}
    \item $A$: time between successive job arrivals (a.k.a ``interarrival time'')
    \item TODO
\end{itemize}

\subsection{Capacity utilization $\rho$}

\begin{equation}
    \rho = \frac{\lambda}{N \mu}
\end{equation}

\begin{itemize}
    \item $\lambda$: job arrival rate
    \item $N$: number of servers
    \item $\mu$: a server's expected service rate, which is the inverse of the average service time $\bar{S}$ \begin{equation}
        \mu = \frac{1}{\bar{S}}
    \end{equation}
\end{itemize}

\subsection{Coefficient of variation of the interarrival time $CV_S$}

\begin{equation}
    CV_S = TODO
\end{equation}

\subsection{Coefficient of variation of the service time $CV_A$}

\begin{equation}
    CV_A = TODO
\end{equation}

\subsection{Expected number of jobs in the queue $L_q$}

\begin{equation}
    L_q = \frac{\rho^{\sqrt{2(N + 1)}}}{1 - \rho} \frac{C_A^2 + C_S^2}{2}
\end{equation}

\noindent\rule[0.5ex]{\linewidth}{1pt}

Midterm exam until here.

\noindent\rule[0.5ex]{\linewidth}{1pt}

\newpage

\section{Exam Cautions}

\begin{itemize}
    \item Draw a supply chain graph clearly showing:\begin{itemize}
        \item the producer
        \item the distributor
        \item the retailer
        \item the customer
    \end{itemize}
    \item Fathom, the ``seller'' and ``buyer'' roles are relative to which stage in the supply chain?
\end{itemize}

\section{Newsvendor Inventory Model}

\textbf{Goal:} Find the optimal order quantity $Q^*$, that only be placed once, based on:

\begin{itemize}
    \item \textbf{purchase cost $c$}: each unit's cost
    \item \textbf{sales price $p$}: each unit's sold price
    \item \textbf{salvage value $s$}: each unit's salvage price; if each unsold unit incurs a disposal cost, the value is \fbox{negative}.
    \item Assume: $p > c > s$
\end{itemize}

\subsection{Two ``costs'' needed to know}

\begin{equation}
    \text{overage cost} = c - s
\end{equation}

Again: remember that $s$ can be negative.

\begin{equation}
    \text{underage cost} = p - c
\end{equation}

\subsection{Critical Fractile}

Expresses the underage cost as a percentage of the combined underage and overage costs.

\begin{equation}
    \text{critical fractile (cf)} = \frac{\text{underage cost}}{\text{underage cost} + \text{overage cost}} = \frac{p - c}{p - s}
\end{equation}

\subsection{Optimal Order Quantity $Q^*$}

\begin{equation}
    Q^* = F^{-1}(\text{cf})
\end{equation}

where $F^{-1}(\alpha)$ denotes inverse cumulative distribution function (CDF).

\subsubsection{Example}

If demand is $\mathcal{N}(\mu, \sigma)$, then $Q^* = \mu + \text{Z-score}(\text{cf}) \cdot \sigma$

\subsection{Multi-product problem with capacity constraint}

Find a single $z$-value that equalizes the fill rate across all products:

\begin{equation}
    z = \frac{K - \sum_{i = 1}^{N} \mu_i}{\sum_{i = 1}^{N} \sigma_i}
\end{equation}

\begin{itemize}
    \item $N$ products that must be ordered
    \item $K$ number units limit across all $N$ products
\end{itemize}

\textbf{Assumption}: all products all have the exact same overage and underage costs. If they differ, the optimal order
quantities have to take into account the other SKU's stocking level to accurately calculate the combined overage and
underage costs.

\section{Bullwhip Effect}

Exists when order volatility (i.e. variance) increases as you move upstream (i.e. farther from the customer / closer to the factory).

\subsection{Causes}

\begin{itemize}
    \item Promotion and volume discounts
    \item Sales incentives
    \item \fbox{No information sharing}
    \item Long lead-times (easily forget about the in-transit inventory)
    \item Anxiety, emotions, distrust, \dots
\end{itemize}

\subsection{Negativities}

\begin{itemize}
    \item High production costs (switch-over, overtime, etc.)
    \item Forcasting challenge
    \item High transportation costs
    \item High inventory but low service level
    \item Hurts trust among business partners
\end{itemize}

\subsection{Solution}

\fbox{Communicate more.}

\section{Base Stock Policy}

\begin{equation}
\begin{split}
    \text{average inventory on-hand} &= \text{expected cycle stock} + \text{expected safety stock} \\
    &= \frac{r \mu}{2} + z \sigma \sqrt{r + L}
\end{split}
\end{equation}

\begin{itemize}
    \item \textbf{$r$ review period}: the interval of time that elapses before an inventory evaluation occurs
    \item \textbf{$L$ replenishment time}: the elapsed time between placing an order and having the ordered unit available to satisfy demand
    \item \textbf{$\alpha$ cycle service level (CSL)}: the probability that the SKUL will have sufficient inventory on-hand to meet all demand in an order cycle
    \item Assume demand follows $\mathcal{N}(\mu, \sigma)$
\end{itemize}

\subsection{If given an average inventory on-hand target of $X$  (time-period-units of supply)}

\begin{equation}
    \text{current average on-hand inventory level} = X \cdot \mu
\end{equation}

where $\mu$ is the mean of the normally distributed demand on the same time-period-unit.

\subsubsection{Impute service level}

\begin{equation}
\begin{split}
    &X \mu = \frac{r \mu}{2} + z \sigma \sqrt{r + L} \\
    \Rightarrow& z = \frac{X \mu - \frac{r \mu}{2}}{\sigma \sqrt{r + L}}\\
    \Rightarrow& \text{service level} = \Phi(z)
\end{split}
\end{equation}

where $\Phi(z)$ is the CDF for the standard normal random variable.

In Excel: \texttt{=NORM.S.DIST(z, TRUE)}

\section{Economic Order Quantity (EOQ) model}

\begin{equation}
    Q^* = \sqrt{\frac{2K\lambda}{h}}
\end{equation}

\begin{itemize}
    \item $K$: fixed order cost incurred every time an order is placed
    \item $\lambda$: demand in units per \underline{year}
    \item $h = Ic$ inventory holding cost: per unit \underline{per year} cost to hold inventory\begin{itemize}
        \item $c$: per unit purchasing cost
        \item $I$: \underline{annual} interest rate
    \end{itemize}
\end{itemize}

If the deterministic demand always arrives in increments of $x$ where EOQ cannot be evenly divided by $x$, update EOQ to
be the nearest multiply of the $x$.

\subsection{Reorder point}

\begin{equation}
    R = L \mu
\end{equation}

\section{Convert the underlying time unit of a normal distribution}

Given \underline{weekly} demand $W \sim \mathcal{N}(\mu_W, \sigma_W)$,\begin{itemize}
    \item Monthly demand $M \sim \mathcal{N}(4 \cdot \mu_W, \sqrt{4} \cdot \sigma_W)$ if $M = (W + W + W + W)$ and $W$ are i.i.d.
    \item Daily demand $D \sim \mathcal{N}(\frac{\mu_W}{5}, \frac{\sigma_W}{\sqrt{5}})$, if $W = (D + D + D + D + D)$ and $D$ are i.i.d.
\end{itemize}

\end{document}