\documentclass[11pt]{article}

\usepackage{sectsty}
\usepackage{graphicx}

% Margins
\topmargin=-0.45in
\evensidemargin=0in
\oddsidemargin=0in
\textwidth=6.5in
\textheight=9.0in
\headsep=0.25in

\title{ 15.761 Introduction to Operations Management }
\author{ Junru Ren }
\date{\today}

\begin{document}
\maketitle	

\section{Little's Law}

\begin{equation}
    L = \lambda W
\end{equation}

\begin{itemize}
    \item $L$: number of jobs in the process
    \item $\lambda$: job arrival rate
    \item $W$: average time each job spend in the process
\end{itemize}

\textbf{Caution}: watch out for unit conversion, especially time-related units.

\section{Queuing Analysis}

\subsection{Setup}

\begin{itemize}
    \item $A$: time between successive job arrivals (a.k.a ``interarrival time'')
    \item TODO
\end{itemize}

\subsection{Capacity utilization $\rho$}

\begin{equation}
    \rho = \frac{\lambda}{N \mu}
\end{equation}

\begin{itemize}
    \item $\lambda$: job arrival rate
    \item $N$: number of servers
    \item $\mu$: a server's expected service rate, which is the inverse of the average service time $\bar{S}$ \begin{equation}
        \mu = \frac{1}{\bar{S}}
    \end{equation}
\end{itemize}

\subsection{Coefficient of variation of the interarrival time $CV_S$}

\begin{equation}
    CV_S = TODO
\end{equation}

\subsection{Coefficient of variation of the service time $CV_A$}

\begin{equation}
    CV_A = TODO
\end{equation}

\subsection{Expected number of jobs in the queue $L_q$}

\begin{equation}
    L_q = \frac{\rho^{\sqrt{2(N + 1)}}}{1 - \rho} \frac{C_A^2 + C_S^2}{2}
\end{equation}

\end{document}